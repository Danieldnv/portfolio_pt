\documentclass{article}
\usepackage{graphicx} % Required for inserting images
\usepackage{amsmath}
\usepackage[a4paper, left=.8in, right=.8in, top=1in, bottom=1in]{geometry}
\usepackage{amssymb}
\title{2º-GQ-Matemática}
\author{Daniel Vital}
\date{November 2024}

\begin{document}

\maketitle

\textbf{Introdução}\\

Variáveis independentes: São as variáveis inseridas e controladas no modelo/cálculo matemático. São as "entradas" (inputs) do sistema\\

\noindent{Variáveis dependentes: variam em função das} independentes.\\

\[
\underbrace{y}_{\substack{\text{Var.}\\\text{Depend.}}}=f\underbrace{(x)}_{\substack{\text{Var.}\\\text{Indep.}}}
\]

\section{Equações diferenciais de primeira ordem}
Geralmente, a solução de um cálculo diferencial é a integral da mesma.\\
A palavra ordem refere-se à mais alta ordem das derivadas  (ou diferenciais) que aparecem na equação diferencial.\\
O grau da equação é dado pelo número que eleva \textbf{exclusivamente a maior derivada}\\
Para que a equação seja \textbf{linear}:
\begin{itemize}
    \item a derivada $\frac{dy}{dt}$ e a variável dependente ($y$) aparecem apenas no primeiro grau
    \item não existe nenhum produto na forma $y\cdot \left( \frac{dy}{dt} \right)$
\end{itemize}


{\noindent{Therefore, em geral, uma equação diferencial linear de primeira ordem assume a seguinte forma:}
\[\frac{dy}{dt} + u(t)\cdot y = w(t)\]\\
Onde: \\
$u, w \; \text{e} \;y$ são funções de $t$\\
Nenhuma restrição é imposta a variável independente $t$\\
$u $ e $w$ podem ser constantes, bem como representar expressões  como $t^2$ e $e^t$\\


quando a função $u$ (o coeficiente da variável dependente $y$) é uma constante, e quando a função $w$ é um termo constante aditivo:\\ reduz-se ao caso especial de uma equação diferencial linear com coeficiente constante e termo constante. Vamos trabalhar, por enquanto, apenas com esse tipo simples de ED.\\

\textbf{O caso homogêneo}. se $u$ e $w$ são funções constantes e $w=0$, fica:\\
\[\frac{dy}{dt} + ay = 0\] ou \[\frac{dy}{dt} \frac{1}{y}=-a\]

\newpage
A solução será:
\[y(t)= \textit{A}e^{-at} \hspace{1cm} \text{[solução geral ; \textit{A} = Cte]}\] ou
\[y(t)=y(0)e^{-at} \hspace{1cm} \text{[solução definida]}\]\\
\section{E.D.O - Resolvendo}


para uma estrutura: $\frac{dy}{dx}+P(x)y=Q(x)$
\\

Ex: \[\frac{dy}{dx}+2y=10\]
\begin{enumerate}
    \item Calcular o fator integrante $\mu (x)$\\
    
    \textbf{Fator integrante $\mu (x)$ = {\large$e^{\int P(x)dx\ }$}}\;\;\;\; $e^{\int 2dx}=e^{2x}$ $\; \xrightarrow{}  \;\;\mu(x)= e^{2x} $
    \item Multiplicar os dois lados pelo fator\\
    
    \[\left(\frac{dy}{dx}+2y \right)\cdot e^{2x}=10\cdot e^{2x}\]
    \[\underbrace{e^{2x}\frac{dy}{dx}+2e^{2x}y}_{(e^{2x}\cdot y)'} = 10e^{2x} \]
    
    \item Reescrever (fator $\times  \;y$)' = ... \;; para facilitar o cálculo\\
    
    (fator $\times \; y )' = (e^{2x}\cdot y)' $\\
    substituindo:\\
    \[\underbrace{(e^{2x}\cdot y)'}_{e^{2x}\frac{dy}{dx}+2e^{2x}y} = 10e^{2x}\]

    \item Calcular a integral dos dois lados da equação
    \[\int(e^{2x}\cdot y)'\ = \int 10e^{2x}\ \]
    $\int(e^{2x}\cdot y)'\ = e^{2x} \cdot y$ 
    \[e^{2x}\cdot y = 10\cdot \frac{1}{2} \cdot e^{2x} + C\]\\
    \[\underbrace{\boxed{y= 5 + e^{-2x}C}}_{\substack{\text{Solução }\\\text{Geral  da E.D.O}}}\]
    
\end{enumerate}
 A solução geral é a função que torna a sentença (E.D.O) verdadeira.
%É diferencial pois, dentro da função há uma incógnita e a derivada dessa incógnita.\\
%A incógnita depende apenas de uma variável independente.

$ e^{-2 \log_e(x)} = \left(e^{\log_e(x)}  \right)^{-2} $

\newpage
\section{Equações diferenciais Homogêneas de 2ª ordem com Coeficientes constantes}
A equação possui a seguinte estrutura
\[
\left\{
\begin{aligned}
y'' = f(x,y,y'') \\
ay(x)''+by(x)'+cy(x) = 0 
\end{aligned}
\right.
\]

Sendo $a, \; b\;, c\; \in\;\mathbb{R}$ (São Ctes) \;\; ; \;\; $a\neq 0$\\


exemplos de estruturas:\\
$y''-y=0$\\
$2y''+3y'+y=0$\\
\\


{\large\noindent{A solução (que tem a mesma propriedade da solução da EDO) é dada por:}}\\
\[y(x)=e^{rx}\]
$e\approx 2,7182$\\

substituindo na equação:
\[a(e^{rx})''+b(e^{rx})'+c(e^{rx})=0\]
\[a\cdot r^2e^{rx}+b\cdot re^{rx}+c\cdot e^{rx}=0\]
\[e^{rx}(ar^2+br+c)=0\]


se $a\cdot b=0\therefore a=0 \;\;\text{ou} \;\;b=0$\\
\indent{porém, $e^{rx} \neq 0$, \;\;\;logo, $(ar^2+br+c)=0$}\\

Para resolver $(ar^2+br+c)=0$, usaremos bhaskara.
Temos que estudar os casos do $\Delta$, pois para cada caso, temos soluções distintas\\
$c_1 \;\text{e} \;c_2$ são constantes arbitrárias. Isto é: Podem assumir qualquer valor, e são determinadas por informações adicionais. Sem condições iniciais, a Eq. diferencial tem infinitas soluções, representadas pelas constantes.\\

\begin{enumerate}
    \item Para o $\Delta > 0\hspace{5cm}$ {\small{Para todo $\Delta >0$, existem duas raízes reais e distintas}}\\
    A solução será:\[\boxed{y(x)=c_1\cdot e^{r_1x}+c_2\cdot e^{r_2x}}\]
    onde $r_1$ e $r_2$ são as raízes\\
    
    $y''-5y'+6y =0\xrightarrow{\;}r^2-5r+6y=0$\\
    $\Delta= b^2-4ac$\\
    $\displaystyle\Delta = (-5)^2-4\cdot 1\cdot 6\hspace{3cm} \frac{-b\pm \sqrt{\Delta}}{2a}=r_1=2$ \;;\; $r_2=3$\\
    $\Delta=1$\\
    $\displaystyle \Delta >0 \;\;\therefore \hspace{3cm} y(x)=c_1\cdot e^{2x}+c_2\cdot e^{3x}$
\newpage
    \item Para o $\Delta =0$\\
    A solução será:
    \[\boxed{y(x)=c_1 e^{rx}+c_2\cdot xe^{rx}}\]

    \item  Para o $\Delta <0$\\
    A solução será:\\
    \[\boxed{y(x)= e^{\alpha x}(c_1\cdot cos(\beta x) + c_2\cdot  sen(\beta x))}\]\\
    sendo: $\alpha$ a parte real \;\; ;\;\; $\beta$ a parte imaginária\\
    Os resultados serão duas raízes complexas conjugadas:\;\;
    \;\;\;\;\;\;\; $r_1=\lambda + \mu i$ \;\; e \;\; $r_2=\lambda - \mu i$\\
    importante: O sinal negativo de $\beta=\pm 1$ é desprezado. Considera-se apenas o valor absoluto (positivo), pois $cos(-x)=cos(x)$ e $sen(-x)=-sen(x)$\\

    Exemplo:
    $y''-2y'+2y=0\therefore \xrightarrow{\text{E.C}} r^2-2r+2=0$\\
    $\Delta = (-2)^2-4\cdot 1\cdot 2=-4$\\

    $\sqrt{-4}=\sqrt{4\cdot(-1)}=\sqrt{4}-\overbrace{\sqrt{-1}}^{i}=2i \hspace{2cm}$\\

    Achado o valor do $\Delta$, podemos resolver com uma diferença apenas na maneira de se escrever:\\

    $\displaystyle \therefore\;\;\;\;\;\frac{-(-2)\pm \sqrt{-4}}{2\cdot 1}=\frac{2 \pm2i}{2}= \underbrace{\frac{2}{2} \pm \frac{2i}{2}}_{\text{atenção}}= 1\pm 1i\;\;\;\;\Rightarrow{} \;\;\;\;\;\;\alpha=1 \;\;;\;\;\beta=1$ \hspace{2.5cm} \text{{\footnotesize{O $\beta$ não inclui ``$i$''}}}\\

    Ou:\\

    $\therefore \displaystyle\;\;\;\;\;\frac{-(-2)\pm \sqrt{-4}}{2\cdot 1}=\frac{2 \pm2i}{2} \Rightarrow{}\;\;\; r_1=\frac{2+2i}{2}=1+1i \;\;\;;\;\;\; r_2=\frac{2-2i}{2}=\underbrace{1-1i}_{\lambda - \mu i}$\\

       
    Independentemente, o resultado será:

    \[\therefore \;\;y(x)= e^{1 x}(c_1\cdot cos1 x + c_2\cdot  sen1 x)\]
\section{Números Complexos}
Todo número real é complexo, mas nem todo complexo é real\\

\noindent{determinaremos $\sqrt{-1}$ como $i$, sendo $i\Rightarrow{}$unidade imaginária de número complexo}\\


exemplo:\\
 $\sqrt{-4}=\sqrt{4\cdot(-1)}=\sqrt{4}\cdot\overbrace{\sqrt{-1}}^{i}=\boxed{2i} \hspace{2cm}$\\
 $i\Rightarrow{}$ unidade imaginária de número complexo\\
 


 \noindent{$\sqrt{-36}=\sqrt{36\cdot(-1)}=\sqrt{36}\cdot\overbrace{\sqrt{-1}}^{i}=\boxed{6i} \hspace{2cm}$}\\

    
    
\end{enumerate}

\newpage
\section{Problema do Valor Inicial}
Para resolver o P.V.I, devemos seguir os seguintes passos:\\
\begin{enumerate}
    \item Resolver a equação diferencial
    \item Calcular a primeira derivada da solução (Apenas na EDO de segunda ordem)
    \item Substituir valores iniciais (Nos casos de EDOs de 2ª ordem, obtêm-se sistemas de equação)
    \item Resolver a equação/equações para achar a(s) contante(s)
    \item Aplicar a(s) constante(s) encontrada(s) na solução geral
\end{enumerate}

\noindent{Exemplo: Resolva o PVI:}\\
\[
\left\{
\begin{aligned}
y''-5y'+6y=0 \\
y(0)=2\\
y'(0)=5
\end{aligned}
\right.
\]
\begin{enumerate}
    \item Achando a solução geral\\
    $y''-5y'+6y \Rightarrow{} r^2-5r+6=0$\\
    $r_1=2\\
     r_2=3$\\
     $\therefore \text{A solução geral será}\;\;\;y(x)=c_1\cdot e^{2x}+c_2\cdot e^{3x}$
     \item Derivando a solução Geral\\
     $y(x)=c_1\cdot e^{2x}+c_2\cdot e^{3x}$\\
     $y'(x)=2c_1e^{2x}+3c_2e^{3x}$
     \item Substituindo os valores iniciais:\\
     $y(0)=2$\\
     $y(x)=c_1\cdot e^{2x}+c_2\cdot e^{3x} \xleftarrow{\;\;\;\;\;\;\;} \text{Se o valor inicial se refere a função não derivada, usamos ela}$\\
     $y(0)=c_1\cdot e^{2\cdot0}+c_2\cdot e^{2\cdot0}$\\
     $y(0)=c_1+c_2$\\
     $y(0)=2 \;\therefore\; \boxed{c_1+c_2=2}$\\

     $y'(0)=5$\\
     $y'(x)=2c_1e^{2x}+3c_2e^{3x}\xleftarrow{\;\;\;\;\;\;}\text{Usando, agora, a função derivada}$\\
     $y'(0)=2c_1e^{2\cdot 0}+3c_2e^{3\cdot0}$\\
     $y'(0)= 2c_1+3c_2$\\
     $y'(0)= 5 \;\therefore\; \boxed{2c_1+3c_2=5}$\\
     \item Montando o sistema com as equações que achamos\\
     \[
\left\{
\begin{aligned}
c_1+c_2=2 \\
2c_1+3c_2=5
\end{aligned}
\right.
\]
$\therefore c_1=1$\;\;;\;\;$c_2=1$

\item Aplicando na solução geral que achamos\\
Sendo $c_1=1$\;\;e\;\;$c_2=1$\\
\[y(x)=c_1\cdot e^{2x}+c_2\cdot e^{3x}\]
\[\therefore \underbrace{\boxed{y(x)=e^{2x}+e^{3x}}}_{\text{Solução do P.V.I}}\]
     
\end{enumerate}


\section{Integrais}
\subsection{Integral Por Substituição}
relembre a regra da cadeia\\
\[\displaystyle[F(g(x))]'=F'(g\left(x\right))\cdot g'(x)\]\\
Integrando essa função:
\[\displaystyle \int [F(g(x))]'dx = \int F'(g\left(x\right))\cdot g'(x)dx\]
\[F(g(x))+C=\int F'(g\left(x\right))\cdot g'(x)dx\]\\
aplicando a substituição:
\[\displaystyle\int F'(\underbrace{g\left(x\right)}_{u})\cdot \underbrace{g'(x)}_{du}dx=F(g(x))+C\]
\[\therefore \int F'(u)\cdot du=F(u)+C\]
{\indent{Ou seja, para aplicar a substituição, precisamos obter uma integral que possua: Uma função e sua respectiva derivada.}}\\

\noindent{Exemplo:}\\
$\displaystyle \int 6x^2(2x^3-1)^{99}dx\;\;\;\;= \;\;\;\;\int du\cdot u^{99}= \frac{u^{100}}{100}+C=\frac{(2x^3-1)^{100}}{100}+C$\\ 

\noindent{} $u=2x^3-1\\
\frac{du}{dx}=6x^2\\
du=6x^2dx$\\

\noindent{}Exemplo 2:\\
$\displaystyle \int \frac{2x}{1+x^2}dx= \;\; \underbrace{\int\frac{du}{u}=ln(u)+C}_{\text{Regra}}=\; ln(1+x^2)+C$\\

\noindent{}$u= 1+x^2\\
\frac{du}{dx}=2x\\
du=2x\cdot dx$\\

\noindent{}Exemplo 3: \hspace{2cm} $\int cos(x+7)dx=cos(u)du= sen(u)+C=\sin(x+7)+C$\\


\noindent{} Exemplo 4:\\
$\displaystyle \int \frac{dx}{(3x-5)^8}= \int \frac{\frac{du}{3}}{u^8}=\int \frac{du}{3}\cdot\frac{1}{u^8}\;=\; \frac{1}{3} \int u^{-8}du\;=\; \frac{1}{3}\cdot\frac{u^{-7}}{-7}+C\;=\; \frac{-(3x-5)^{-7}}{21}+C$\\
$u=3x-5\\
\frac{du}{dx}=3\\
du=3dx\\
\frac{du}{3}=dx$\\
\newpage
\section{Integral por partes}
Podemos lembrar a regra do produto, nas derivadas, para fixar:\\
Prestar atenção ao escolher ``$dv$" e ``$u$" para simplificar a resolução.\\

\[(u\cdot v)'=du \cdot v\;+\;u\cdot dv\]
{\indent{integrando os dois lados:}}\\
\[u\cdot v=\int du\cdot v\;+\; \int u\cdot dv\]
{\indent{Reorganizando a equação:}}\\
\[\boxed{\int u\cdot dv= u\cdot v \;-\; \int v\cdot du}\]\\

\noindent{}Exemplo:\\
$\int xe^x\,dx\\ \int u\cdot dv= u\cdot v \;-\; \int v\cdot du$\\
$ \int xe^xdx= x\cdot e^x-\int e^x\cdot1\,dx$\\

O resultado será:
\[\therefore\int xe^xdx= xe^x \;-\; e^x+C\]

\noindent{$u=x\\
\frac{du}{dx}=1\\
du=dx\\$}\\
$dv = e^xdx\\
v=e^x\xleftarrow{\;\;\;}$ Se a derivada é $e^x$, qual a função que ao ser derivada, é $e^x?$\\













\end{document}
